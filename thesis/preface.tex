\chapter*{Introduction}
\addcontentsline{toc}{chapter}{Introduction}


Clustering analysis as a method of unsupervised learning technique has become well spread over many fields of science nowadays. Originated in anthropology, clustering analysis (clustering) is the task of organizing set of objects into such sub-sets that they show a kind of similarity. 

Clustering has a great foundation in machine learning but can be recently found in the field of molecular biology as well~\cite{Nugent2010}. Although this field has many opportunities for using machine learning algorithms to advance the research~\cite{btaa091}, potential of clustering in molecular biology is restricted by the size of a working dataset. Reasonably computable dataset sizes vary from 10K to~100K of objects due to the polynomial complexity of used algorithms. One way to increase this size is incorporating computing power of graphical processing units (GPUs) into the algorithms. For example, \citet{andrade2013g} showed that GPU implementation of DBSCAN \footnote{Density-based spatial clustering of applications with noise} can be over 100-times faster than its sequential CPU counterpart. With this possible performance increase, researchers can use datasets with millions of objects. Moreover, \citet{hua2017mgupgma} proposed the~UPGMA\footnote{Unweighted Pair Group Method with Arithmetic Mean} implementation that utilized multiple GPUs. This shows that clustering algorithms have a big potential for a measurable performance increase when implemented on a GPU device.

To this day, there is a gap in GPU alternatives of this algorithms. In the present thesis, we implement a hierarchical clustering with the Ma\-ha\-la\-no\-bis-based distance on a single GPU and test its performance. In this manner, the thesis firstly describes different variations of clustering algorithms with addition to some possible optimizations. Next, CUDA --- a GPU programming platform --- is introduced. Finally, experiments are performed on the implementation to summarize the results.
