\chapter*{Introduction}
\addcontentsline{toc}{chapter}{Introduction}


Clustering analysis as a method of unsupervised learning technique has become well spread over many fields of science nowadays. Originated in anthropology, clustering analysis (clustering) is the task of organizing set of objects into such sub-sets that they show a kind of similarity. 

Clustering has a great foundation in machine learning but can be easily found among many fields like pattern recognition~\cite{pare} or image processing~\cite{sathya2011image}. It can be found in the field of molecular biology as well~\cite{Nugent2010}. Although this field has many opportunities for using machine learning algorithms to advance the research~\cite{btaa091}, a potential of clustering in molecular biology is restricted by the size of a working dataset. Reasonably computable dataset sizes vary from 10K to~100K of objects due to the polynomial complexity of used algorithms. One way to increase this size is incorporating computing power of graphical processing units (GPUs) into the algorithms. For example, \citet{andrade2013g} showed that GPU implementation of DBSCAN \footnote{Density-Based Spatial Clustering of Applications with Noise} can be over 100-times faster than its sequential CPU counterpart. With this possible performance increase, researchers can use datasets with millions of objects. Moreover, \citet{hua2017mgupgma} proposed the~UPGMA\footnote{Unweighted Pair Group Method with Arithmetic Mean} implementation that utilized multiple GPUs. This shows that clustering algorithms have a big potential for a measurable performance increase when implemented on a GPU device.

To this day, there is a gap in GPU alternatives of this algorithms. The current thesis aims to implement GPU accelerated hierarchical clustering algorithm with the Mahalanobis distance. In this manner, we aim to explore possible agglomerative clustering algorithms and the data structures they use. According to their space and time complexity and ability to be parallelized, we choose the most suitable algorithm to be implemented utilizing a heavy computing power of a GPU. Our goal is to increase the size of a reasonably computable cytometry datasets by an order of a magnitude (to millions of objects) while preserving the same computational time. This contribution would open new possibilities to progress in the research of a cell cytometry data.

In the manner of fulfilling the claimed aims, the structure of the thesis is as follows. The first chapter describes different variations of clustering algorithms with addition to some possible optimizations. Next, the second chapter focuses on the actual GPU implementation. Hence, an overview of CUDA --- a GPU programming platform --- is introduced. This is followed by the description of the GPU Mahalanobis-based hierarchical clustering implementation. Finally, experiments are performed on the implementation to summarize the results.

With the utilization of the heavy GPU parallel properties using CUDA platform, the proposed implementation achieves an exceptional performance increase in clustering of single-point datasets compared with the serial implementation.