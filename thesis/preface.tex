\chapter*{Introduction}
\addcontentsline{toc}{chapter}{Introduction}

Clustering is a commonly used technique for simplifying the work with complex data\-sets.
The main principle is to represent a dataset composed of a huge number of elements by a simplified (and significantly smaller) set of element groups which are commonly called clusters.
There are many various ways of constructing the clustering algorithms because each discipline typically requires a different kind of cluster similarity, it manipulates completely different data (texts, sequences, vectors) and it has different expectations about the organizing of the resulting data.

Clustering is commonly used to analyze datasets originating in single-cell cytometry~\cite{shapiro2005practical}, where grouping the cells by similar measured features often corresponds to creating clusters of all individual biological types of the cells. This greatly simplifies a data analysis and a distinction of different cell types. Many customized approaches for clustering cytometry data have been developed, including FlowSOM~\cite{van2015flowsom}, PhenoGraph~\cite{levine2015data}, SPADE~\cite{qiu2011extracting} or FlowGrid~\cite{ye2019ultrafast}. This thesis focuses on agglomerative hierarchical clustering with the Mahalanobis metric, originally developed by \citet{fivser2012detection} for the purpose of monitoring minimal residual decease in patients with leukemia. 

The Mahalanobis clustering is, however, seriously restricted by the performance and complexity of the current implementation, which can cluster tens of thousands of cells on a common hardware.
The computation of these clusters is time demanding and consumes a lot of computer memory. Spe\-cifically for the Mahalanobis clustering, the maximum size of a dataset that can be processed on common hardware varies at about 100,000 cells. As a result, the performance of the Mahalanobis clustering is unsuitable for processing data from modern cytometers used in current experiments, which often produce datasets of more than several million cells.

The primary goal of this thesis is to research possibilities of accelerating the Mahalanobis clustering on a GPU, develop the implementation of the clustering accelerated on a GPU and measure its results. Many clustering algorithms have been already ported to a GPU device, eg.~DBSCAN~\cite{andrade2013g}, UPGMA~\cite{hua2017mgupgma}, etc.. The Mahalanobis clustering is extremely suitable for cytometry because it naturally forms elliptical clusters. A GPU-accelerated implementation that could improve the performance on large datasets has not been developed yet.

The thesis first discusses the hierarchical clustering (section~\ref{sec01:hierarch_clust}) and variations of algorithms that can be used (section~\ref{sec01:hca}). Then, an overview of the programming for the GPU is introduced (section~\ref{sec02:cuda}). Using this information, the thesis designs a parallelized algorithm of MHCA which can be run on a GPU; therefore, it is significantly faster. The combination of simple metric spaces with Minkowski distance specialization and enormous GPU throughput allows for a major memory requirements reduction. The total acceleration gain of the implementation varies from 20 times to 5000 times. The quality and speed of the result is demonstrated on flow and mass cytometry datasets. We hope that the results of the thesis will be possible to package and use for biologically relevant purposes.


