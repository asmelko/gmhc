\chapter*{Introduction}
\addcontentsline{toc}{chapter}{Introduction}

Clustering is a commonly used technique to simplify a work with complex data\-sets.
The main principle is to represent a dataset composed of a huge number of elements by a simplified (and significantly smaller) set of element groups which are commonly called clusters.
Clustering is a rather branched topic because each discipline typically requires a different kind of cluster similarity, it manipulates completely different data (texts, sequences, vectors) and it has different expectations about the organizing of the resulting data.

Clustering is commonly used to analyze data in cytometry because it groups cells of the same type. This greatly simplifies a data analysis and a distinction of different cell types. There is a lot of approaches of clustering cytometry data~\cite{lo2008automated,ge2012flowpeaks,zeng2007feature}. This thesis focuses on agglomerative hierarchical clustering with the Mahalanobis metrics~\cite{mahalanobis1936generalized} that can be used for monitoring minimal residual decease in patients with leukemia~\cite{fivser2012detection}. The use of clustering in cytometry has a fundamental problem. The computation of these clusters is time demanding and consumes a lot of computer memory. Specifically for the Mahalanobis clustering, the maximum size of a reasonably computable dataset varies at about 100,000 cells. As the time and technology advances and modern cytometers produce more data, the need to cluster millions and billions of cells arises.

The primary goal of this thesis is to research possibilities of accelerating the Mahalanobis clustering on a GPU, develop an implementation and measure the results. Many clustering algorithms are already ported to a GPU device, eg.~DBSCAN~\cite{andrade2013g}, UPGMA~\cite{hua2017mgupgma}, etc.; the Mahalanobis clustering is extremely suitable for cytometry because of its natural creation of elliptical clusters but the fast implementation has not been developed yet.

The thesis first discusses the hierarchical clustering (section~\ref{sec01:hierarch_clust}) and variations of algorithms that can be used (section~\ref{sec01:hca}). Then, an overview of the programming for the GPU is introduced (section~\ref{sec02:cuda}). Using this information, the thesis designs a parallelized algorithm of MHCA which can be run on a GPU; therefore, it is significantly faster. The combination of simple metric spaces with Minkowski distance specialization and enormous GPU throughput allows for a major memory requirements reduction. The total acceleration gain of the implementation varies from 20x to 5000x. The quality and speed of the result is demonstrated on flow and mass cytometry datasets. We hope that the results of the thesis will be possible to package and use for biologically relevant purposes.


