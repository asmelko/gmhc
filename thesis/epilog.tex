\chapter*{Conclusion}
\addcontentsline{toc}{chapter}{Conclusion}

Over the past years, implementing of general clustering algorithms using GPUs to accelerate a related research has become an expanding topic. Still, there are many algorithm that are supported only on CPUs, which brings great opportunities for increasing the performance. In the current thesis we proposed and tested an implementation of Mahalanobis-based hierarchical clustering accelerated on a single GPU. 

For the proposed implementation, we used CUDA platform to communicate with a GPU. We employed high throughput operations like shuffle instructions and utilized a device memory hierarchy using shared and constant memory. With the stated features, we were able to achieve great performance for the resulting implementation.

The proposed MHCA implementation provides performance increase of up to 5400-fold in single-point datasets and 20-fold in apriori datasets. Moreover, it requires only linear space with respect to the dataset size; this is a property which is difficult to find in different implementations.

With such performance increase, this implementation is able to extend the size of reasonably computable datasets from hundreds of thousands to small millions of points. It can provide opportunities to advance research in the field of cell cytometry by decreasing time required for computations as well as enabling to compute larger datasets.

The experiments showed a space for improvements in the proposed implementation as well. The computing of distance between small clusters may be too simple and can propagate a creation of unwanted clusters. This weakness can be addressed by implementing a different strategy for early stages of the algorithm.

The work can be further expanded by implementing the Full Mahalanobis distance (see def.~\ref{def01:fmd}). It can provide more precise dissimilarity measure, which can lead to a better clustering results. Naturally, it would come for the price of greater overall time complexity. 